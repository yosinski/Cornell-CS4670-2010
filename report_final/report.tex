\section{Introduction} % short intro

\edit{Cooper}



\section{Related Work} % related work, with references to papers, web pages

We used some data from Yale \cite{GeBeKr01}...

\edit{Cooper: copy our references slide}



\section{Approach} % technical description including algorithm

\edit{Cooper: intro do approach}

\subsection{Detection}

\subsection{Preprocessing}

\edit{Cooper, aka face detection, resizing, grayscale..} 

\subsubsection{Database Generation}
\subsubsection{Eigenvector Generation}



\subsection{Eigenvector Nearest Neighbor}

\edit{Cooper}


\subsection{Fisher Linear Discriminant}

\edit{Jason}



\subsection{Computation Considerations for Mobiles Devices}

\edit{Cooper}

\subsubsection{Load-Time}
\subsubsection{Picture}



\section{Results} % experimental results

\subsection{Face Detection}


Some results using our Haar wavelet detector are shown in
\figref{small_face.jpg} through \figref{three_faces.jpg}.

\figvarp{small_face.jpg}{.60}{Haar wavelet detection of a face taking up
  a small fraction of the image.}{}

\figvarp{small_face_sideways.jpg}{.60}{Our Haar wavelet detector tends
  not to work if the faces are rotated more than 30 degrees.}{}

\figvarp{three_faces.jpg}{.60}{Detection of multiple faces at different
  scales in a single image works well.}{}

\edit{clean this up...}

\subsection{Eigenvector creation}

\edit{pull from slide}

\subsection{Eigenvector Nearest Neighbor results on Yale Data set}

\edit{Cooper}

\subsection{Fisher results on Yale Data set}

\edit{Jason}

\subsection{Fisher results on phone/computer images}

\edit{Jason}


\section{Discussion} % discussion of results, strengths/weaknesses, what worked, what didn't

\edit{Jason}


\section{Future Work} % future work and what you would do if you had more time

\edit{Jason}


\section{Conclusion}

\edit{Cooper}










\section{Results}




\section{Fisher Faces}

Equations from wikipedia:

\begin{equation}
S=\frac{\sigma_{between}^2}{\sigma_{within}^2}= \frac{(\vec w \cdot \vec \mu_{y=1} - \vec w \cdot \vec \mu_{y=0})^2}{\vec w^T \Sigma_{y=1} \vec w + \vec w^T \Sigma_{y=0} \vec w} = \frac{(\vec w \cdot (\vec \mu_{y=1} - \vec \mu_{y=0}))^2}{\vec w^T (\Sigma_{y=0}+\Sigma_{y=1}) \vec w}
\end{equation}

\begin{equation}
\vec w = (\Sigma_{y=0}+\Sigma_{y=1})^{-1}(\vec \mu_{y=1} - \vec \mu_{y=0})
\end{equation}

\begin{equation}
\vec w = (\Sigma_{y=0} + \Sigma_{y=1} + \epsilon I)^{-1}(\vec \mu_{y=1} - \vec \mu_{y=0})
\end{equation}


\begin{equation}
s_{ratio} = \frac{s_{test} - s_{\mu_0}}{s_{\mu_1} - s_{\mu_0}}
\end{equation}
