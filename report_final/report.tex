%\section{N/A} % title, team members
\section{Introduction} % short intro
\section{Related Work} % related work, with references to papers, web pages
\section{Approach} % technical description including algorithm
\subsection{Preprocessing}
\subsubsection{Database Generation}
\subsubsection{Eigenvector Generation}
\subsection{Onboard Computation}
\subsubsection{Load-Time}
Fisher Faces
\subsubsection{Picture}
\section{Results} % experimental results
\section{Discussion} % discussion of results, strengths/weaknesses, what worked, what didn't
\section{Future Work} % future work and what you would do if you had more time
\section{Conclusion}


%%%%%%%%%%%%%%%%%%%%%%%%%%%%%%%
% Midterm report below here (old work):
%%%%%%%%%%%%%%%%%%%%%%%%%%%%%%%

\section{Progress}

Our tasks and our progress on each are shown below.

\newcommand{\prog}[1]{\textit{\\
#1}}

\begin{enumerate}
\item Set up a server for collaboration and running required
  server-side code.  \prog{Done. We set up a server and are using Trac
    and Git for
    collaboration\footnote{\code{http://cs4670.yosinski.com/}}.  We're
    currently running the vision code locally, but will transition
    parts of it to the server as the need arises.}
\item Code and demonstrate basic face detection.  \prog{Done.  We have
  implemented basic face detection using a Haar wavelet detector from
  OpenCV; see Results section.}
\item Code and demonstrate basic facial feature extraction.
  \prog{Investigated options, coding not yet started.}
\item Integrate detection and description code. \prog{Not started.}
\item Match detected features to database of features. \prog{Not started.}
\item Compile for phone image capture and processing.  \prog{Not started.}
\item Bonus: build database using Facebook images.  \prog{Not started.}
\end{enumerate}



\section{Results}

Some results using our Haar wavelet detector are shown in
\figref{small_face.jpg} through \figref{three_faces.jpg}.

\figvarp{small_face.jpg}{.60}{Haar wavelet detection of a face taking up
  a small fraction of the image.}{}

\figvarp{small_face_sideways.jpg}{.60}{Our Haar wavelet detector tends
  not to work if the faces are rotated more than 30 degrees.}{}

\figvarp{three_faces.jpg}{.60}{Detection of multiple faces at different
  scales in a single image works well.}{}



\section{Fisher Faces}

Equations from wikipedia:

\begin{equation}
S=\frac{\sigma_{between}^2}{\sigma_{within}^2}= \frac{(\vec w \cdot \vec \mu_{y=1} - \vec w \cdot \vec \mu_{y=0})^2}{\vec w^T \Sigma_{y=1} \vec w + \vec w^T \Sigma_{y=0} \vec w} = \frac{(\vec w \cdot (\vec \mu_{y=1} - \vec \mu_{y=0}))^2}{\vec w^T (\Sigma_{y=0}+\Sigma_{y=1}) \vec w}
\end{equation}

\begin{equation}
\vec w = (\Sigma_{y=0}+\Sigma_{y=1})^{-1}(\vec \mu_{y=1} - \vec \mu_{y=0})
\end{equation}

\begin{equation}
\vec w = (\Sigma_{y=0} + \Sigma_{y=1} + \epsilon I)^{-1}(\vec \mu_{y=1} - \vec \mu_{y=0})
\end{equation}


\begin{equation}
s_{ratio} = \frac{s_{test} - s_{\mu_0}}{s_{\mu_1} - s_{\mu_0}}
\end{equation}



%Possible reference for Yale database:
% @Article{GeBeKr01,
%  author =  ``Georghiades, A.S. and Belhumeur, P.N. and Kriegman, D.J.'',
%  title =   ``From Few to Many: Illumination Cone Models for Face Recognition under
%               Variable Lighting and Pose'',
%  journal = ``IEEE Trans. Pattern Anal. Mach. Intelligence'',
%  year =  2001,
%  volume = 23,
%  number = 6,
%  pages= ``643-660''}
% 
% 
  
